\section{Twitter}

Twitter es una red social que permite a los usuarios enviar y leer mensajes cortos de un máximo de 140 caracteres, fue lanzada por una empresa de diez jóvenes en San Francisco en julio de 2006 . Posee una interfaz web, donde se muestran los propios mensajes del usuario(tweets). Los usuarios pueden indicar si desean que sus tweets sean públicos (es decir que puedan ser vistos por cualquier usuario de internet) o reservados a un ambiente privado (que sólo pueden ser visualizados por los seguidores del usuario en Twitter), éstos mensajes se pueden publicar mediante Twitter.com, mensajería de texto, mensajería instantánea o mediante clientes que facilitan el servicio.

Los usuarios de Twitter siguen a otros o se siguen. A diferencia de la mayoría de los sitios de redes sociales en línea, como Facebook o MySpace, la relación de seguir y ser seguido no requiere reciprocidad. 

Una práctica habitual en Twitter es responder a un tweet mediante un lenguaje simbólico de etiquetas bien definido: RT significa re-tweet (o replicar el mensaje en cuestión), '@' seguido de la dirección del identificador del usuario, se refiere a un usuario en específico y un '\#' seguido de una palabra representan un \emph{hashtag} (Una forma de etiquetado de metadatos, los \emph{hashtags} proporcionan un medio de agrupar este tipo de mensajes, ya que uno puede buscar el \emph{hashtag} y obtener el conjunto de mensajes que lo contienen). El mecanismo de re-tweet permite a los usuarios difundir la información de su elección más allá del alcance de lo seguidores del autor original.

Twitter actualmente cuenta con cerca de 316 millones de usuarios activos mensuales, donde se producen cerca de 500 millones de tweets al día \cite{cifrasTwitter}, mientras que el 2012 presentaba un ritmo de crecimiento impresionante de 11 nuevas cuentas Twitter por segundo \cite{infografiasLabs}.

Twitter se clasifica como un servicio de microblogging, permitiendo a los usuarios enviar actualizaciones de texto breves o micromedios como fotografías o clips de vídeo, poseen una importante componente de tiempo real. Los usuarios escriben mensajes de Twitter varias veces en un sólo día. Pueden saber lo que otros están haciendo o pensando de manera inmediata, los usuarios vuelven repetidamente al sitio y comprueban para ver lo que otros están haciendo. De esta manera surgen numerosos informes de diversos eventos no sólo de la vida privada de cada usuario, sino también de eventos públicos compartidos o vividos por muchos usuarios.

En quizás el primer estudio del uso de Twitter, Java \cite{JavaEtAl:07} identificó tres categorías principales de los usuarios de Twitter: Las \emph{fuentes de información}, los \emph{amigos} y los \emph{solicitantes de información}.
\emph{Las fuentes de información} publican noticias y tienden a tener una gran base de seguidores, estas fuentes pueden ser personas físicas o servicios automatizados. Los \emph{amigos} es una categoría amplia que abarca la mayoría de los usuarios, incluyendo la familia, compañeros de trabajo y extraños. Por último, los \emph{solicitantes de información} tienden a ser usuarios que pueden crear información, pero que rara vez son seguidos por otros usuarios de manera regular.

En el estudio realizado por Java \cite{JavaEtAl:07} también se identificaron varias categorías de tipos de usos de Twitter, incluyendo la categoría de charla cotidiana, donde los usuarios discuten acontecimientos de sus vidas personales o de sus pensamientos actuales, intercambian información o enlaces, noticias de la presentación de informes, los cuales incluyen comentarios sobre la actualidad. El  más relevante uso es la intención del usuario de conversación. Tomando la aparición de la señal \"\@\" como un indicador, Java encontró que el 21\% de los usuarios en su estudio utilizó Twitter para este propósito.

\subsection{Trending topic}

Twitter sigue y destaca las frases, palabras y \emph{hashtags} que más a menudo sean mencionados en la red social y cataloga éstos con el nombre de \emph{trending topics}(tendencia o tema del momento).
Twitter muestra una lista de los diez \emph{trending topics} del momento en una barra lateral situada a la derecha de la página principal de cada usuario de forma predeterminada, a menos que se disponga lo contrario. La gran repercusión que ha tenido en la prensa ha provocado que esta expresión sea utilizada también para denominar un tema de gran interés Algunos ejemplos de trending topics fueron por ejemplo \emph{la muerte de Michael Jackson}, \emph{la muerte de Amy Winehouse}, \emph{los finales de la UEFA Champion League} o \emph{la aparición del nuevo Iphone}. Debido a las limitancias de esta red social las temáticas son expresadas por frases principales ejemplo \#QEPDMickaelJackson.

\subsection{Twitter en Chile}

El uso de Twitter en Chile es impresionante, contando con 5 millones de cuentas activas de Chile (posicionándose el 2012 entre los \emph{top ten} en su uso a nivel mundial). El tiempo estimado de navegación y uso fluctuaría entre 7.7 horas diarias. \cite{rankingChileSemiocast}

Según el perfil de uso de Twitter en Chile desarrollado en \cite{udpperfiluso}. Respecto a la frecuencia con la que los usuarios utilizan Twitter 76\% la utilizan varias veces al día, 16\% al menos una vez al día y el resto al menos una vez por semana. Los horarios de mayor uso de twitter se encuentra desde las 10 de la mañana en adelante (un 60\% de los usuarios comienzan a utilizar la red a esta hora) alcanzando su peak de uso entre las 19 y 22 horas con un 73\% de los usuarios.

Respecto a la intención de uso 45\% de los usuarios lo utiliza para mantenerse informado, 25\% para debatir y expresar opinión, 16\% por entretenimiento y el restante 14\% se reparte entre diversas razones: mantener vínculos profesionales, mantener contacto con amigos y conocidos, para hacer nuevos amigos entre otras.