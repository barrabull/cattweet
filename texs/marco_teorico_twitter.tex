\section{Twitter}

Twitter es una red social que permite a los usuarios y usuarias enviar y leer mensajes cortos de un máximo de 140 caracteres, fue lanzada por una empresa de diez jóvenes en San Francisco en julio de 2006. Posee una interfaz web donde se muestran los propios mensajes del usuario(tweets). Los usuarios pueden indicar si desean que sus tweets sean públicos (que puedan ser vistos por cualquier usuario o usuaria de internet) o reservados a un ambiente privado (que sólo pueden ser visualizados por los seguidores y seguidoras del usuario o usuaria) y publicarlos mediante la plataforma web, mensajes de texto, la aplicación móvil o mediante clientes que facilitan el servicio.

Los usuarios y usuarias de Twitter siguen a otros y otras y son seguidos por otros y otras. A diferencia de la mayoría de los sitios de redes sociales en línea, como Facebook o MySpace, la relación de seguir y ser seguido no requiere reciprocidad. 

Una práctica habitual en Twitter es responder a un tweet mediante un lenguaje simbólico de etiquetas bien definido: RT significa re-tweet (o replicar el mensaje en cuestión), "@" seguido de la dirección del identificador del usuario o usuaria se refiere a un usuario o usuaria en específico y un '\#' seguido de una palabra representan un \emph{hashtag} (Una forma de etiquetado de metadatos, los \emph{hashtags} proporcionan un medio de agrupar este tipo de mensajes, ya que uno puede buscar el \emph{hashtag} y obtener el conjunto de mensajes que lo contienen). El mecanismo de re-tweet permite a los usuarios y usuarias difundir la información de su elección más allá del alcance de lo seguidores del autor o autora original.

Twitter actualmente cuenta con cerca de 316 millones de usuarios y usuarias activas mensuales, donde se producen cerca de 500 millones de tweets al día \cite{cifrasTwitter} y presentando en el 2012 presentaba el impresionante ritmo de crecimiento de 11 nuevas cuentas de Twitter por segundo \cite{infografiasLabs}.

Twitter se clasifica como un servicio de microblogging, permitiendo a los usuarios y usuarias enviar actualizaciones de texto breves o micromedios como fotografías o clips de vídeo y posee una importante componente de tiempo real. Los usuarios y usuarias escriben mensajes de Twitter varias veces en un sólo día. Pueden saber lo que otros y otras están haciendo o pensando de manera inmediata, los usuarios y usuarias vuelven repetidamente al sitio y comprueban para ver lo que otros y otras están haciendo. De esta manera surgen numerosos informes de diversos eventos no sólo de la vida privada de cada usuario o usuaria, sino también de eventos públicos compartidos o vividos por muchos usuarios y usuarias.

En un estudio del uso de Twitter \cite{JavaEtAl:07}, Java  identifica tres categorías principales de los usuarios y usuarias de Twitter: Las \emph{fuentes de información}, los \emph{amigos} y los \emph{solicitantes de información}.
\emph{Las fuentes de información} publican noticias y tienden a tener una gran base de seguidores, estas fuentes pueden ser personas físicas o servicios automatizados. Los \emph{amigos} es una categoría amplia que abarca a la mayoría de los usuarios y usuarias, incluyendo la familia, compañeros y compañeras de trabajo y extraños o extrañas. Por último, los \emph{solicitantes de información} tienden a ser usuarios o usuarias que pueden crear información, pero que rara vez son seguidos por otros usuarios y usuarias de manera regular.

Java \cite{JavaEtAl:07} también identifica varias categorías de tipos de usos de Twitter incluyendo la categoría de charla cotidiana, donde los usuarios y usuarias discuten acontecimientos de sus vidas personales o de sus pensamientos actuales, intercambian información o enlaces, noticias las que incluyen comentarios sobre la actualidad. El uso más relevante es la intención de conversación, considerando la aparición de la señal "\@" como un indicador, Java determina que el 21\% de los usuarios y usuarias utilizaron Twitter para este propósito.

\subsection{Trending topic}

Twitter hace seguimiento de	 las frases, palabras y \emph{hashtags} que más a menudo son mencionados en la red social y los cataloga con el nombre de \emph{trending topic} (tendencia o tema del momento).
 Esta característica ha tenido gran repercusión en la prensa logrando que sea utilizado también para denominar un tema de gran interés. Algunos ejemplos de \emph{trending topics} en Twitter fueron por ejemplo \emph{la muerte de Michael Jackson}, \emph{la muerte de Amy Winehouse}, \emph{los finales de la UEFA Champion League} o \emph{la aparición del nuevo Iphone}. Debido a las limitancias del largo de los textos en Twitter las temáticas son expresadas por frases principales como por ejemplo \#QEPDMickaelJackson.
 
 Twitter muestra de manera predeterminada una lista de los diez \emph{trending topics} del momento en una barra lateral situada a la derecha de la página principal de cada usuario y usuaria, a menos que se disponga lo contrario.

\subsection{Twitter en Chile}

 Twitter cuenta con 5 millones de cuentas activas en Chile (posicionándose el 2012 entre los \emph{top ten} en su uso a nivel mundial). Contando con un tiempo estimado de navegación y uso que fluctúa entre 7.7 horas diarias según un estudio realizado por Semiocast SAS \cite{rankingChileSemiocast}.

Según el perfil de uso de Twitter en Chile desarrollado en \cite{udpperfiluso} 76\% las y los chilenos utilizan Twitter varias veces al día, 16\% al menos utilizan Twitter una vez al día y el resto al menos una vez por semana. Mientras que los horarios de mayor uso se encuentran desde las 10:00 hrs. en adelante (un 60\% de los usuarios y usuarias comienzan a utilizar la red a esta hora) alcanzando su \emph{peak} de uso entre las 19:00 hrs. y 22:00 hrs. con un 73\% de las usuarias y usuarios.

Respecto a la intención de uso, 45\% de los usuarios y usuarias utilizan Twitter para mantenerse informado, 25\% para debatir y expresar opiniones, 16\% por entretenimiento y el restante 14\% se reparte entre diversas razones: mantener vínculos profesionales, mantener contacto con amigos y conocidos, para hacer nuevos amigos entre otras.