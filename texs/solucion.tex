Existe un creciente desconfianza por parte de la ciudadanía respecto a la repercusiones reales del proceso de \emph{gatekeeping} en las noticias publicadas por los medios de prensa tradicionales.

La irrupción de Internet y las redes sociales en la vida cotidiana han modificado variados esquemas de comunicación y formas de relación
entre las personas, entre ellos, los medios de prensa y la forma que tienen las y los ciudadanos para informarse de los eventos que ocurren en su entorno. Tambini en  \footnote{http://wallblog.co.uk/2013/03/05/how-twitter-won-the-social-media-battle-for-journalism/\#ixzz2kBQ2RFcQ } señala ``El papel de \emph{gatekeeper} de los medios de prensa tradicionales se ha debilitado y las personas construyen su propia estructura editorial eligiendo que siguen en los medios de comunicación basados en la recomendación. Las redes sociales también difieren en el sentido de que tienen un compromiso con la universalidad y apertura, que es más importante para ellos que los periódicos estén ahí" 

Twitter debido a las múltiples características que posee referente a su capacidad de propagación, acceso y espontaneidad se presenta como una importante red social que hoy revoluciona la industria periodística mundial, obligando a las grandes cadenas a integrarles dentro de su proceso periodística. Hughes, un corresponsal muy influyente y famoso de la BBC, dice: ``Los medios sociales permiten que consiga mucho más acerca de la historia. Hay periodistas y otras personas en el mismo lugar del suceso que reportan los eventos en tiempo real, de manera que cuando una historia en realidad aparece en la televisión, a menudo ya la he visto través de los medios sociales". Hughes declara, que al igual que una gran cantidad de periodistas, recolecta inicialmente una gran cantidad de noticias desde Twitter. Otros informes dicen que la cifra de periodistas recibiendo historias y eventos de Twitter es de aproximadamente 50\%. Hughes dice que el 80\% de su recopilación de noticias la obtiene en Twitter y sólo el 20\% de otras fuentes \cite{RePEc:ehl:lserod:59881}.

Considerando el interés creciente de adquirir información de eventos noticiosos desde sus fuentes directas evitando los procesos de \emph{gatekeeper} sumado a las potencialidades que posee Twitter referente a esta misma temática se busca diseñar e implementar una herramienta que permita comunicar los reportes en Twitter de un determinando evento, privilegiando aquellos reportes con menor cantidad de intermediadores y posibles \emph{gatekeeper}, siendo estos, las y los observadores y las y los reportadores directos del suceso.

\section{Objetivos}

\subsection{Objetivo principal}

\begin{itemize}
\item Desarrollar un algoritmo computacional que permita recoger tweets que reporten un evento, priorizando 
los tweets geoposicionados cercano al lugar de ocurrencia del evento, para generar un relato temporal
referente a dicho evento que será presentado mediante una interfaz web.
\end{itemize}

\subsection{Objetivos Secundarios}
\begin{itemize}
\item Analizar trabajos previos y herramientas creadas con anterioridad para encontrar la forma adecuada y más conveniente de proceder a la construcción de esta herramienta.
\item Dotar al público interesado en informarse sobre eventos noticiosos de una herramienta de reporte de eventos que minimiza el filtrado y tratamiento editorial de contenidos.
\item Desarrollar una interfaz web que sea clara y fácil de usar por el usuario.
\end{itemize}
