\subsection{Clasificación de tweets y usuarios de Twitter}

Debido a la gran cantidad de usuarias y usuarios que forman hoy en día la red social de Twitter, la clasificación tanto de cuentas como de contenidos se ha convertido en una creciente materia de estudio para la comunidad científica. En el siguiente apartado se revisan los trabajos realizados referentes a ésta temática.

\subsubsection{Clasificación de usuarios}
   
     Pennacchiotti y Popescu en \cite{PennacchiottiP11} desarrollan una clasificación del perfil de las cuentas considerando principalmente cuatro aspectos diferentes:

    \begin{enumerate}
        \item \emph{Perfil del usuario}: Información básica de la cuenta como el nombre de usuario, la locación, una pequeña biografía además del número de seguidores y seguidoras, el número de personas a las que sigue  y el número de tweets. 
        
        \item \emph{Comportamiento para escribir tweets}:  Conjunto de métricas de las interacciones
        existente entre la red social y el usuario o usuaria: el número promedio de tweets por minuto, número de respuestas, entre otras. 
        
        \item \emph{Contenido lingüístico de los tweets}: Encapsula los temas principales de interés del usuario, así como su uso de léxico, este análisis mediante el uso de \emph{palabras prototípicas} \footnote{ El análisis se realiza mediante el uso de las cuales son expresiones típicas de las personas pertenecientes a una clase específica, así como frases relacionadas a intereses típicos de dicha clase} permite la clasificación de los usuarios y usuarias según su estilo de escritura tales como textos oficiales, blogs, conversaciones o traducciones.
       
        \item \emph{Información de Twitter}: Estas características exploran las relaciones sociales establecidas por
        la o el usuario con los demás que él o ella siguen, a quien le responde o que personas re-tweetea.          Pennacchiotti y Popescu indican que existe la idea intuitiva de que las personas pertenecientes a una clase son más propensos a seguir las cuentas de ciertas personas y a responder a ellas (por ejemplo, las jóvenes pueden tender a responder a la cuenta de Justin Bieber).
       
    \end{enumerate}

	        Respecto al análisis del \emph{Perfil del usuario} Pennacchiotti y Popescu tras analizar un corpus de
	        14 millones de cuentas, identifican que sólo el 48\% provee una biografía corta y 80\% una ubicación, de cuya información se intentaron determinar el género del usuario o usuaria y su etnicidad pero los resultados obtenidos fueron de muy baja calidad. 
	        
	        Mediante una muestra 15000 cuentas de forma aleatoria y un conjunto de editores y editoras a quienes se pidió identificar la enticidad y el género a partir de la imagen del avatar de Twitter. Se obtiene que menos del 50\% de las imágenes se correlaciona de manera clara con alguna etnia mientras que el 57\% se correspondía con algún género. Se identifica también que las imágenes podían ser engañosas: en el 20\% de los casos la imagen no corresponde al dueño o dueña de la cuenta sino de una celebridad o de otra persona.
	        
	         Con esta información estadística el estudio concluye que los campos del perfil del usuario no contienen suficiente información ni de buena calidad para ser utilizada para una clasificación.
	        %En trabajo anteriores Cheng en \cite{Cheng:2010:YYT:1871437.1871535} estableció que sólo el 26\% de los usuariosde Twitter reportan su ubicación como una ciudad específica, el resto provee locaciones generales (estados o paises) o lugares imaginarios. 
	        
	        
	        %En \cite{JavaEtAl:07} han ratificado percepciones intuitivas como: Cuando los usuarios postean rara vez pero tienen una gran cantidad de seguidores tienden a ser un buscador de información. Mientras que los usuarios que poseen URL's en sus propios tweets son proveedores de información. Rao en \cite{Rao:2010:CLU:1871985.1871993} sugiere que el estudio de estas componentes no es útil para la mayoria de las tareas de clasificación y que queda incluida en los rasgos lingüísticos.
	        
	        
	        Referente al \emph{Contenido lingüístico de los tweets} se realiza un análisis de los hashtag prototípicos, basados en la hipótesis de que si los usuarios o usuarias de una clase están interesados en los mismos temas, los temas más populares de esta clase se pueden encontrar mediante la recopilación de estadísticas sobre hashtags usados. A modo de síntesis se representa a un usuario mediante el conjunto de palabras de sus tweets y mediante éstas se intenta clasificar a dicho usuario.
	               
	         También se realiza un análisis de sentimientos a los distintos tweets, pues frente a un tema en particular dos clases pueden expresar distintos sentimientos. Por ejemplo: dos grupos políticos pueden expresar opiniones positivas o negativas de una figura pública dependiendo generalmentee si es del mismo sector político a ellos.  Para este proposito, se recoge un conjunto de palabras para las clases observadas en este estudio sobre las que el usuario particular tiene una opinión global que en su mayoría no es compartida por otra clase diferente.

    Las clases específicas estudiadas se refieren a: afiliación política, clasificación de si es o no seguidor de Starbucks y si los usuarios  y usuarias pertenecen o no a la etnia afroamericana. 
    
    %Los resultados obtenidos indican que para los dos primeras clases es posible conseguir un buen resultado con una considerable precisión, mientras que para el tercero es más compleja la tarea.

    Para determinar la afiliación política se concluye que las mejores características para su determinación son
    lingüísticas y de perfil. Siendo desconsiderable el aporte que entregan a esta determinación la información de Twitter y el comportamiento en "tweetear".

    `Para determinar si un usuario o usuaria pertenece a la clase \emph{seguidor o seguidora potencial de Starbucks} se identifica que  la \emph{información del perfil} y \emph{análisis lingüístico} son las  características más útiles para este objetivo.

    Se concluye además que la relación entre las y los seguidores y amigas y amigos es también una carácteristica relevante por sí sola para determinar la clase de seguidores o seguidoras, pues sugiere que los aficionados de Starbucks son los que siguen a los demás más que seguirse entre sí pues en su mayoría son los solicitantes de información (probablemente gente en busca de ofertas y cupones).

    Si bien el uso del léxico determina en este experimento con bastante precisión a la etnia afro-americana por sus variadas expresiones, cabe destacar que dichas expresiones ligúisticas han sido ampliamente adoptadas por otros grupos haciendo visible que esta característica posee una clara limitante en su aplicación. Los mejores resultados para determinar la etnia de un usuario es mediante la combinación del léxico y si siguen a celebridades a fines. Se encontró también que a tarea de clasificación puede ser ayudada por información del perfil.
  
 \subsubsection{Ranking de enlaces compartidos en Twitter}
  
  En \cite{Dong:2010:TEI:1772690.1772725} Dong utiliza como fuente de información fresca los contenidos y enlaces compartidos en Twitter para búsquedas web en tiempo real. Dong identifica que en vista de la investigación de  Hughes en \cite{hughes2009twitter} donde concluye que ante un evento inesperado, los tweets contienen más información relevante que en una situación normal (y tienen un enfoque más de \emph{broadcasting}), es posible considerar Twitter como una buena fuente de información en tiempo real en base a cuatro oportunidades:
  
  \begin{itemize}
	\item Los enlaces compartidos pueden corresponder a noticias o no (permitiendo recoger información sobre los enlaces que no son noticias y mejorar los resultados de la búsqueda).
	\item Los enlaces difundidos son publicados en base a las distintas prioridades personales de las y los usuarios, lo que aporta un interesante grado de diversidad.
	\item La red de Twitter permite realizar mediciones de autoridad a las y los creadores de tweets.
	\item Los tweets cuentan con metadatos relacionados que permiten clasificarlos e inferir en base a su relevancia.
  \end{itemize}
  
 Se consideran filtros en el procesamiento de enlaces referente al spam \emph{ Corresponden a enlaces propagados en Twitter referente a un producto o marca sin contenido relevante} basados en heurísticas a fines (como filtrar enlaces twitteadas por el mismo usuario más de dos veces o solo twitteadas por un mismo usuario).
 
 Referentes a los enlaces y los tweets que los contenían se consideran las siguientes carácterísticas:
  %Puesto que el sistema de Twitter se separa en dos componentes muy distintos, estos dos componentes que son: un sistema de publicación y otro de subscripción. El sistema de publicación permite detectar nuevas URL's mientras que el sistema de subscripción permite recoger información sobre la calidad de estas nuevas URL's.
  
\begin{itemize}
	\item \textbf{Características textuales}: Se considera que las palabras que acompañan a una URL en un tweet pueden entregar información relevante sobre ésta. Principalmente se realiza un conteo de estas palabras y la cantidad de repeticiones existentes para todos los tweets analizados, generándose un conjunto de pares de palabras y URL's relevantes (Similar al análisis del \emph{Contenido lingüistico de los tweets} considerado en el estudio analizado anteriormente \cite{PennacchiottiP11}).
	
	\item \textbf{Características de redes sociales}: Se aplica el concepto de autoridad a los usuarios de Twitter, vinculados mediante las relaciones de re-tweet entre ellos.
	
	\item \textbf{Otras características}: Se define un conjunto de diez características adicionales, muchas de las cuales consideran el ranking establecido en base a la autoridad de los usuarios. Estas se dividen en tres grupos:
		\begin{itemize}
		\item{Referentes al promedio del conjunto de usuarios que publicaron la URL}: número promedio de \emph{followers} de las y los usuarios, número promedio de tweets de las y los usuarios, numero promedio de las y los usuarios que retweetean cantidad de tweets que contienen la URL, promedio de usuarias y usuarios que responden los tweets que contienen la URL, promedio del número de usuarias y usuarios a las que siguen, promedio del ranking de autodidad (definido anteriormente).
		\item{Referentes al usuario que inicialmente twitteó la URL}: Se asume autoridad para el primer emisor de la URL en base a algunos criterios como número de followers, número de tweets, número de usuarios que realizaron retweet, número de respuestas, número de personas a las que sigue.
		\item{Referentes al usuario que posee mayor ranking de autoridad} Las características observadas son número de followers, número de tweets, número de usuarios que realizaron retweet, número de respuestas, número de personas a las que sigue y el ranking de autoridad.
		\end{itemize}
\end{itemize}
	
	El ranking se basa en una máquina de aprendizaje de ranking (MLR) que considera las características anteriores con distintas ponderaciones de importancia calificándolas con un factor en una escala de [0,100] (de forma descendente, las más importantes son: La característica de repetición de la URL en los distintos tweeets y su relación con las palabras relacionadas, número de seguidoras y seguidores del usuario con mayor grado de autoridad, número de usuarias y usuarios que re-tweetan la URL al usuario con mayor grado de autoridad, número promedio de usuarios y usuarias que re-tweetean los tweets que contienen la URL y número promedio de usuarios y usuarias que siguen a quienes han twitteado la URL). Los resultados se clasificaron en base a una tupla (query,URL,$t_query$) con grados de relevancia y se constrataron con la opinión de cinco expertos en cinco grados de clasificación: perfecto, excelente, bueno, justo y malo. Se agrega además un sistema de etiquetas de tiempo debido al interés especial de esta dimensión del problema, donde los tweets fueron clasificados en dos grandes categorías: sin sensibilidad de tiempo y con senbilidad de tiempo con las subcategorías: \emph{reciente}, \emph{de alguna manera reciente}, \emph{de alguna manera fuera de fecha} y \emph{totalmente fuera de fecha}.
	
	Si los documentos cuentan son de la categoría: \emph{sin sensibilidad de tiempo}, \emph{reciente} o \emph{de alguna manera reciente} al momento de su clasificación preservan su clasificación, en cambio, cuando son \emph{sensibles al tiempo} se utiliza un sistema de descenso de categoría:
	
	\begin{itemize}
		\item \emph{Descenso de un grado}: Si el resultado es \emph{de alguna manera fuera de fecha} se realiza un descuento de un grado (por ejemplo de excelente a bueno).
		\item \emph{Descenso de dos grados}: Si el resultado es \emph{totalmente fuera de fecha} se realiza un descuento de dos grados (por ejemplo de excelente a malo).
	\end{itemize}
	
	Los resultados obtenidos fueron contrastados con expertos bajo el criterio de que sólo evaluaran una query como "con contenidos relevantes de última hora" si el sistema arrojaba al menos un documento creado en las últimas 24 horas con contenido relevante. Se obtuvo que el 91,7\% de las consultas fueron clasificados de esta manera y que una gran cantidad de URL's poseen la calidad de enlaces frescos, lo que hace concluir que casi no existen documentos obsoletos en las URL's de Twitter. 
	
	Se identifica también que el porcentaje de edades de los enlaces clasificados como \emph{perfectas} o \emph{excelentes} es más alta comparativamente que los enlaces \emph{regulares}, mientras que las clasificadas como \emph{justo} y \emph{mala} son más bajas que las URLs \emph{regulares}, demostrando de esta manera que las características de Twitter pueden mejorar el rendimiento de un sistema de \emph{ranking} sensible al tiempo.  
	
	Finalmente se concluye que los enlaces propagados en Twitter son útiles para mejorar potencialmente la clasificación de consultas de búsqueda sensibles al tiempo.
	
\subsubsection{Mecanismo de Ranking en Twitter como foro}

	 En \cite{DasSarma:2010:RMT:1718487.1718491} se presenta un análisis de un sistema de ranking que podría ser perfectamente aplicado a Twitter por sus características, sugiriendo una arquitectura genérica de un sistema de clasificación para diversos items (tweets, post u otro) con la intención de conseguir la mejor clasificación posible con el menor esfuerzo implicado.

Las características deseadas para el sistema de ranking propuesto son:

	\begin{itemize}
		\item \emph{Precisión del Ranking}: El ranking debe ser preciso aún cuando no sea posible re-evaluar nuevamente todos los items implicados.
		
		\item \emph{Revisión del ancho de banda}: El ranking debe converger al orden correcto, dentro del nivel de precisión deseado rápidamente con una pequeña cantidad de retroalimentación por artículo.
		
		\item \emph{Baja Latencia}: Los usuarios y usuarias no deben esperar mucho tiempo para recibir un estimado de sus puntuaciones o clasificaciones actualizadas.
		
		\item \emph{Equidad}: Los items deben ser tratados igualmente con respecto a la clasificación y la revisión.
	\end{itemize}

	Debido a que la distribución de probabilidad con la cual son evaluados los diversos elementos, su evaluación depende principalmente del orden en que son presentados, se intenta contrarrestar este efecto mediante el diseño de formas explicitas de evaluación en este trabajo. Bajo esta misma perspectiva, se elige
	el método de evaluación comparativo entre dos elementos a modo de torneo, el cual funciona de la siguiente manera: cuando el usuario o usuaria envía un nuevo elemento (tweet, comentario o post) se le muestra un par de otros items (seleccionados dependiendo la distribución de torneo sorteada al azar) entre los cuales selecciona el mejor, dichas clasificaciones son recogidas y evaluadas (estas evaluaciones cuentan con una mejor estimación de rango entre más evaluaciones poseen dichos elementos).
	
	Se concluye que el sistema de \emph{ranking} planteado posee mejor rendimiento -evaluado en base a las carácterísticas deseadas- que un sistema de calificación individual (como el sistema de clasificación por estrellas de plataformas como Netflix).
	
\subsubsection{Twitter para la recomedación de noticias}

	Muchos de los sistemas actuales de recomendación, se basan en las preferencias personales de los usuarios, en \cite{Phelan:2009:UTR:1639714.1639794} utiliza Twitter y fuentes RSS para este cometido.
	
	El sistema se compone de tres grandes componentes:
	\begin{itemize}
		\item \textit{Componente Web}: Encargada de reunir la información disponible del usuario en RSS y en Twitter para recoger las preferencias de la usuaria o usuario.
		\item \textit{Index Lucene\footnote{Lucene es una biblioteca de búsqueda de texto completo extremadamente rica y poderosa escrita en Java}}: Responsable de la indexación y la minería de la información obtenida.
		\item \textit{Recomendación}: Encargado de generar una lista clasificada de historias RSS basado en la co-ocurrencia de términos populares dentro de los post y gustos expresados en Twitter.
	\end{itemize}
	
	La indexación se genera principalmente al transformar las palabras contenidas en los últimos tweets del usuario en una matriz de co-ocurrencia $M$ ($M_{ij}$ representa la cantidad de ocurrencias que posee la palabra $j$ en
	el tweet $i$), las co-ocurrencias mayores se relacionan con el conjuntos de artículos que las contienen, tras sumar esta cantidad se le asigna un puntaje basado en la sumatoria de co-ocurrencias evaluadas. De esta forma, se crea un ranking de temas y artículos a sugerir (ordenados entre estos mismos según su grado de relevancia). 
	
	Por otra parte el usuario puede elegir tres estrategias distintas de recomendación: 
	
	\begin{itemize}
		\item \textit{Ranking Público}: Basado en el análisis de los tweets públicos del timeline del usuario.
		\item \textit{Ranking de las y los amigos}: Basado en el análisis de los tweets públicos de los timelines de los amigos y amigas del usuario.
		\item \textit{Ranking de Contenido}: No utiliza Twitter, sólo considera las 100 mayores ocurrencias de palabras del análisis de las RSS.
	\end{itemize}
   
   Se evalúa la aceptación de las recomendaciones realizadas por este sistema mediante el criterio de un pequeño grupo de 10 participantes, en un período de 5 días y se cuantifica la cantidad de clicks recibidos por noticia recomendada por el sistema. Cada participante configuró el sistema con su información correspondiente.

	Al analizar los resultados se obtiene una clara diferencia en el comportamiento de las y los usuarios cuando se comparan las estrategias basadas en Twitter a la basada en el contenido predeterminado. Se observa, por ejemplo, que para la primera prueba se recibe una media 8,3 y 10,4 \emph{clicks} por cada usuario en comparación con sólo el 5,8 \emph{clicks} por usuario en la estrategia basada en el contenido; expresando un relativo aumento de entre el 30\% y el 45\% para las estrategia basada en Twitter.

	Se observa también que el \emph{ranking} de mayor uso de las recomendaciones basadas es el \emph{ranking de las y los amigos} en comparación a las recomendaciones del \emph{ranking público}, resultado que se contradice con un cuestionario posterior realizado, donde un 67\% de las y los usuarios indican su preferencia por \emph{ranking público} mientras que sólo el 22\% señala su preferencia por el \emph{ranking de las y los amigos}. Ninguno de los participantes se muestra partidario de la estrategia de \emph{ranking de contenido} y un 11\% no saben cual estrategia prefieren.

\subsubsection{Clasificación de tweets orientada al usuario: Un enfoque de filtrado para microblogs}

%En  se plantea la relevancia de un filtrado de contenido en Twitter, ya que debido a su enorme volumen de información carece de valor sin no es presentada de manera adecuada. 

Ibrahim y Bruce en \cite{conf/cikm/UysalC11} plantean que la acción de re-tweetear para clasificar a los usuarios posee un gran valor: re-tweetar incluye leer el tweet, decidir que vale la pena compartir y luego actuar sobre ella, por lo cual es posible considerar el re-tweet como una señal explícita de que la usuaria o usuario considera el tweet como información relevante. Basado en esta apreciación, se busca dar respuesta a la interrogante  ¿Es posible clasificar a los usuarios basado en si re-tweetean un tweet específico?

El objetivo de este trabajo es clasificar la cuenta de un usuario para mostrar los tweets entrantes en un orden descendente en función de su probabilidad de ser retweeteado por el mismo. Una clasificación efectiva ayudará al usuario a encontrar los tweets potencialmente más interesantes. Como experimento preliminar, se utiliza un árbol de decisión (J48) (implementado en WEKA) para determinar la precisión con la que se puede clasificar los tweets como re-tweetable o no para un usuario específico. Se entrena el clasificador se utilizan cuatro grupos de rasgos:

\begin{itemize}
	\item \textit{Basado en el autor}: Se refiere a características deducidas a partir del perfil del usuario, relacionadas a qué tan activo es el usuario o usuaria y la autoridad que posee.: ¿Es el usuario un autor de élite \footnote{\emph{élite} local si su cantidad de seguidores esta en el rango de 10K-50K followers, \emph{élite global} si su cantidad de followers es mayor a 50K y \emph{usuario ordinario} si el usuario tiene de 10 a 1000 seguidores o personas a las que sigue, escribe entre 1 a 200 tweets por semana y ha twitteado más de 10 veces.} ?, cantidad de followers, cantidad de personas a las que sigue, cantidad de tweets,  edad del perfil \footnote{cantidad de días desde la creación de la cuenta}, tasa de tweets, cantidad de favoritos, ¿tiene descripción?, ¿su idioma es el inglés?. 
	
	\item \textit{Basado en los tweets}: Se refiere a características sintácticas del tweet, algunas de éstas dan implicaciones sobre que tan bien está escrito el tweet: la categoría, la audiencia y la popularidad del tweet.: La puntuación TF-IDF\footnote{TF-IDF es una estadística numérica que se pretende reflejar la importancia de una palabra es un documento, colección o corpus}), ¿contiene hashtag? ¿contiene urls?¿Menciona a otros usuarios? ¿utiliza comillas para citar? ¿escribe el mismo carácter tres veces seguidas (por ejemplo: hoolaaa)?¿utiliza emoticonos?, la cantidad de re-tweet del tweet y el largo promedio de los tweets.
	
	\item \textit{Basado en el contenido}: Se refiere a las características relativas al contenido del tweet: novedad del tweet (distancia coseno de términos de los otros tweets que aparecen en el timeline en la última semana) y lo inesperado del tweet (distancia mínima de la distancia coseno de términos con los demás tweets del autor).
	
	\item \textit{Basado en el usuario}: Características relacionadas a la cuenta del usuario: Del tweet re-tweteado ¿sigue el usuario al autor? ¿utiliza el hashtag del tópico relacionado en otros tweets? ¿se comparten enlaces con el tweet? ¿se menciona al usuario en el tweet en cuestión?
\end{itemize}

Para el conjunto de entrenamiento se realizan dos clases: los tweets re-tweteados y los que no son re-tweeteados. Los resultados obtenidos señalan que ningún grupo de características arrojó resultados satisfactorios por si solos, entre las cuales la mejor característica de todas es la tasa de tweets (cantidad de tweets por semana). De las características basadas en los tweets las más valiosas fueron: si el tweet fue re-tweteado y la puntuación tf-idf. De las características basadas en  el usuario se considera útiles las características: "¿es el autor del tweet una persona a la que el usuario sigue?","¿se compartieron enlaces con el usuario?", "¿se utilizó el hashtag del tópico relacionado en otros tweets?" y "¿se menciona al usuario en el tweet en cuestión?".

\subsubsection{TURank: Clasificación del usuarios de Twitter basado en el análisis de un grafo usuario-tweet}

 En \cite{Yamaguchi:2010:TTU:1991336.1991364} se aborda el problema de identificar los usuarios con autoridad en una red de microblogging como es Twitter. Se entiende por un usuario con autoridad aquellos usuarios que frecuentemente suben información a la red, que es considerada relevante y es propagada de forma rápida y amplia.
 
 Debido a la gran cantidad de información que se encuentra en esta red social es preciso identificar los usuarios con autoridad que dinamizan y aportan con información relevante, su identificación podría tener múltiples impactos en el manejo y categorización de la información. Muchos trabajos enfocan su análisis en la estructura de vínculos de seguidores que mantienen los usuarios, sin embargo, la mayoría de las y los usuarios siguen de vuelta al usuario que comenzó a seguirlos por un acto de cortesía formal, por lo cual se considera poco relevante. En este trabajo se plantea un algoritmo que considera la puntuación de autoridad de los usuarios de Twitter considerando un grafo de relaciones sociales además del análisis del flujo de tweets entre los usuarios (mediante el re-tweet).

En Twitter, un usuario sigue a otro, si es probable que el usuario transmita información útil incluso sin garantías. En muchos casos, los seguidores no dejan de seguirlo incluso si resulta que no transmite  más información útil. Esto ocurre porque los usuarios no recuerdan a todos las usuarias y los usuarios a los que están siguiendo y debido al gran número de cuentas a las que el usuario sigue (de 100 a más de 1000 en muchos casos). Incluso si un usuario tiene una gran cantidad de seguidores no implica que estos sean frecuentemente re-tweetados.

En cuanto a la propagación de la información, cabe señalar que los re-tweets tienen distintas características dependiendo del objetivo que esta acción tenga, si es un re-tweet conversacional tiene sólo re-tweets de los implicados en dicha conversación, muy por el contrario si es un re-tweet de difusión, este tiene un  gran número de re-tweet y se propagan ampliamente. Por lo anterior, no es suficiente sólo contabilizar la cantidad de re-tweets para medir la autoridad de una usuaria o usuario, ya que los re-tweets conversacionales son menos relevantes para la puntuación de autoridad en cuestión. 

%Se consideran además los re-tweet realizados por usuarios con autoridad, pues es probable que estos retweets son más útiles que retweets de usuarios que no son autoridad.

Se utiliza un tipo de grafo direccional llamado \emph{esquema de transferencia de autoridad} donde se representan el dominio del discurso y los flujos respectivos de autoridad donde cada nodo representa todo el conjunto de elementos de destinos y las aristas todo el conjunto de relaciones o trasferencias que puede ocurrir entre ellos. Este gráfico es evaluado mediante \emph{ObjectRank} \footnote{ObjectRank es una extensión de PageRank\cite{ilprints422} para medir la importancia de los objetos en una base de datos teniendo en cuenta el tipo de aristas del grafo así como el tipo de nodo}.

La evaluación de los distintos usuarios y usuarias se basa en tres observaciones:

\begin{itemize}
	\item Una usuaria o usuario seguido por muchas autoridades probablemente es también una autoridad.
	\item Un tweet re-tweeteado por alguna autoridad probablemente sea un tweet útil.
	\item Una usuaria o usuario que posee muchos tweets útiles es probablemente una autoridad.
\end{itemize} 

Basado en estas observaciones fue construido un grafo donde los nodos representan usuarios y tweets mientras que las aristas representan las relaciones entre usuarios y tweets. Este grafo \emph{usuario-tweet} permite comprender cómo se propaga la información entre los usuarios mediante el re-tweet. Posteriormente se realiza un análisis de enlaces y se calcula la autoridad de los usuarios utilizando \emph{ObjectRank}. Las pruebas realizadas fueron contrastadas por un conjunto de expertos, y se obtuvo como conclusión que el número de followers y la cantidad de re-tweet no son suficientes para determinar si un usuario posee o no autoridad. El cuantificador de re-tweet tiende a extraer a los usuarios que utilizan el re-tweet para las conversaciones con el fin de especificar al usuario al cual se dirigen, en estos casos los tweets no trasmiten información útil.

Se demuestra que a pesar de su estructura simple, el grafo planteado describe las relaciones usuario-tweet lo suficientemente bien, representando adecuadamente las cadenas de re-tweet y múltiples re-tweets realizados por el mismo usuario o usuaria. Por otra parte, el ranking planteado es un eficaz sistema de puntuación para evaluar la autoridad de los usuarios de Twitter ya que evalúa a las usuarias y usuarios que no son seguidos por muchos usuarias y usuarios, pero sus tweets son re-tweeteados muchas veces, con una mayor posición mientras que a las usuarias y usuarios cuyos tweets no se re-tweetean, incluso si tienen un gran número de seguidores, se les asigna una peor la posición en el \emph{ranking}. Por último, a aquellas usuarias y usuarios en los cuales la mayoría de sus tweets son conversaciones, son evaluados como completamente inútiles por el algoritmo. 
