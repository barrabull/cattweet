\subsubsection{Panel de enlaces}

Esta funcionalidad recoge todos los enlaces contenidos en los tweets del tópico con la intención de crear una sección de \emph{bibliografía multimedia} de un tópico permitiendo acceder de manera directa a enlaces externos que permiten extender la información del tópico aportada por los distintos tweets, así como acceder a enlaces de distintas posturas permitiendo entre otras cosas contraponer la forma en que presentan la información, profundidad, etc.

Los enlaces son ordenados en base a su tweet de origen, del más reciente al menos reciente y luego del que posee más re-tweets al que posee menos.

\begin{algorithm}
	\caption{Obtención de enlaces externos contenidos en los tweets}\label{getURLAlgo}
	\begin{algorithmic}[H]
		\Function{GetUrlsByTweets}{tweets}
		tweets $\gets$ ordenarRelevancia(tweets);
		%\State Obtener lista de tweets del topico ordenados del más reciente al menos reciente, del que posee más re-tweets al que posee menos
		\For{ tweet in tweets}
			\If{hasURL(tweet)}
				\State urls.append(tweet.url)
			\EndIf
		\EndFor
		\For{url in urls}
			\State url.titulo, url.link, url.imagen $\gets$ getInformationURL(url)
		\EndFor 
		\EndFunction

	\end{algorithmic}
\end{algorithm}

Debido a la gran versatilidad del origen que poseen los distintos enlaces compartidos, existen dificultades en su recolección por distintos motivos como: URL incorrectas o inexistentes, textos que no es posible manipular, etc.

Para capturar los datos del enlace, se utiliza la librería lxml \ref{itm:lxml} de Python, que tras navegar por la estructura del documento HTML se extrae el título, el enlace completo y una imagen representativa. Las condiciones mínimas que se aplica a cada uno de estos datos, es que el titulo tenga más de 10 caracteres, que la imagen posea el metatag \emph{image} de Opengraph \cite{openGraphProtocol} y que el enlace no retorne error 404 (página no encontrada).


	
	   