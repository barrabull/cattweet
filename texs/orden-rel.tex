\subsubsection{Orden de relevancia}\label{subsubsec:orden-rel} 	

El objetivo de este \emph{ranking}, es generar un mecanismo que ordene los contenidos en base a su relevancia, entendida como el grado de utilidad de un tweet para informar al usuario respecto al evento en cuestión.

Para su diseño se considera la cantidad de re-tweets con que cuenta el tweet basado en el razonamiento abordado en \cite{conf/cikm/UysalC11} donde se considera que la acción de re-tweet resumen en un solo indicador, la importancia que le atribuyen sus lectores al contenido del tweet y lo replican, porque lo consideran relevante y en el estudio de la intención realizado en \cite{Yamaguchi:2010:TTU:1991336.1991364} referente a que si la intención del re-tweet es difundir (intención buscada por el prototipo, tweets que busquen informar y difundir sobre el tópico), es bastante probable que posea una gran cantidad de re-tweet, no así los tweets conversacionales .
 
 En el diseño también se considera la fecha de emisión del tweet, implementando un sistema de tres clasificaciones basado en el ranking desarrollado en \cite{Dong:2010:TEI:1772690.1772725} con la diferencia que se utilizan tres grados de clasificación y no cinco.

El mecanismo de descenso implementado considera que un tweet esta \emph{fuera de la fecha} cuando la fecha de emisión del tweet es 4 días antes que la fecha de emisión del tweet más reciente del conjunto de tweets y considera que un tweet esta totalmente \emph{fuera de fecha} cuando la fecha de emisión del tweet es 10 o más días antes que la fecha de emisión del tweet más reciente del conjunto. Para los tweets \emph{fuera de fecha} se aplica un descenso de una categoría, mientras que para los tweets \emph{totalmente fuera de fecha} se aplica un descenso de dos categorías.

\begin{algorithm}[H]
	\caption{Orden Relevancia}\label{OrdenRel}
	\begin{algorithmic}[1]
		\Function{ordenRelevancia}{}
		\State maxRT $\gets$ getMaxRT()
		\State minRT $\gets$ getMinRT()
		\State fechaMasReciente $gets$ getFechaMasReciente()
		\State firstClass, secondClass, thirdClass $gets$ dividirConjuntoPorRT(tweets)
		
		\For{ tweet in firstClass}
		\State deltaFecha $\gets$ diff(fechaMasReciente, tweet.fecha)
		
		\If{ deltaFecha $\geqslant$ 4 dias}
			\State tweet.clase $\gets$ tweet.clase - 1
			\ElsIf { deltaFecha $\geqslant$ 10 dias}
				\State tweet.clase $\gets$ tweet.clase - 2
		\EndIf
		
		\EndFor
		
		\For{ tweet in secondClass}
		\State deltaFecha $\gets$ diff(fechaMasReciente, tweet.fecha)
			\If{ deltaFecha $\geqslant$ 4 dias}
				\State tweet.clase $\gets$ tweet.clase - 1
			\EndIf
		\EndFor	
		\EndFunction
	\end{algorithmic}
\end{algorithm}

%\emph{Resultados}
	
%	Para realizar experimentos con este orden se seleccionaron X tweets de forma aleatoria de un topico determinado, estos tweets fueron presentados de forma aleatoria en un formulario de papel en donde por cada tweet se presentaba:
	
%	\begin{itemize}
%		\item Texto del tweet
%		\item Fecha del tweet
%	\end{itemize}
	
%	Las instrucciones del formulario para cada evaluador experto indicaba: \emph{Ordene los siguientes textos de tweets, colocando un numero en la linea ubicada a la derecha de cada texto, según su relevancia (grado de utilidad del tweet para informar al usuario respecto al evento en cuestión) }. Los resultados obtenidos fueron los siguientes:
	
%	\input{result-orden-rel.tex}
	