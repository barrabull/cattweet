\subsection{Conclusiones técnicas}

El presente trabajo abarca aspectos teóricos y prácticos respecto a la problemática
de informarse sobre un evento noticioso y profundiza en el desarrollo de una algoritmo computacional que recolecta información desde Twitter y la presenta en una interfaz web, con varias alternativas de presentación de la información con distintas vistas de ordenamiento del contenido y enlaces recogidos. Los tweets se presentan de manera descendente en una línea de tiempo con enlaces directos (tanto al perfil del autor como a los distintos) tweets facilitando el acceso directo a la fuente original, entregando una experiencia simple e intuitiva.

En la sección \ref{sec:definicion_problema} se aborda la discusión de las miradas existentes al proceso de \emph{gatekeeping}. Una parte importante de autores sostienen que estos filtros son intencionados y diseñados con el objetivo de moldear la realidad que se trasmite mientras que el otro grupo sostiene que éstos no son intencionales sino necesarios y de origen netamente operativo. En este trabajo, aún cuando se evitaron aplicar deliberadamente filtros de contenidos (de naturaleza editorial o ideológica), fue necesario - debido a la gran cantidad de datos disponibles en Twitter -  generar clasificaciones y selecciones para extraer y presentar la información relevante (explicados en la sección \label{chap:propuesta_cap}.), sin estos tratamientos (debido a la gran cantidad de tweets recogidos) éstos carecen de valor.

En la sección \ref{chap:estadodelarte} se analizan las distintas investigaciones y herramientas de fines similares que tuvieran relación con aspectos abarcados en este trabajo. De manera general, es posible verificar a través de la creciente  cantidad de estudios sobre Twitter el ascendente interés de la comunidad científica sobre esta red social. Muchos de los trabajos revisados abordan de distintas maneras la categorización de la información para dar solución a la problemática del valor de la información ante a los grandes volúmenes de Twitter. Para la clasificación de usuarios las estrategias variaban en la consideración de distintas características relacionadas a los tweets o a los usuarios en la plataforma, las más efectivas fueron aplicadas completa o parcialmente para el diseño de este prototipo. Por su parte la revisión de las herramientas existentes, permite evidenciar la emergente industria de las aplicaciones y servicios que buscan recoger y presentar los contenidos de las redes sociales con variados objetivos entre los que se encuentran: monitoreo de opiniones sobre una marca, lectura resumida de las publicaciones de los contactos de un usuario en las redes sociales, mostrar información de tendencia, búsqueda de tweets o comprobación de la veracidad de la información.

Respecto al geoposicionamiento de los usuarios, aspecto fundamental para el diseño del prototipo, fue posible evidenciar las dificultades respecto a este tema: sólo cerca del 20\% de los usuarios completan el campo \emph{ubicación} del perfil de Twitter, el resto lo completan con lugares muy generales o ficticios. Esta dificultad acotó las expectativas del prototipo de poder privilegiar las fuentes geográficamente más cercanas al lugar del hecho noticioso (desarrollado mediante el ranking geográfico explicado en \ref{ordenGeoIndex}). El método implementado en este trabajo logra relacionar el 17,54\% de los usuarios totales con una provincia específica de Chile mientras que para el caso de prueba los tweets con localización corresponden al 26\% de los tweets del tópico. Existen variadas técnicas utilizadas para mejorar este relacionamiento en los diversos estudios analizados  \cite{Cheng:2010:YYT:1871437.1871535} \cite{McGee:2011:GST:2063576.2063959} pero debido a su complejidad se pretende abordar como desarrollo futuro. 



%Otra de las dificultades enfrentadas fueron los límites para la obtención de datos impuestos por la API de Twitter, estas limitaciones afectaron en el periodo de tiempo necesario para recolectar datos. 

%Una de las dificultades enfrentadas más relevantes fue la falta de estudios relacionados al uso de Twitter y al periodismo en las redes sociales enfocados en el territorio Chile. Esta dificultad fue enfrentada mediante el uso de bibliografía relacionada a otros medios de difusión disponibles en blogs y notas de prensa y en experimentos sobre el conjunto de datos recogidos.


Respecto a los resultados abordados en \ref{chap:discusion} se verifica que Twitter es una fuente rica en información para la elaboración de reportes de un hecho noticioso (desde la cual se pueden filtrar efectivamente los tweets no relacionados), no sólo limitada a las noticias redactadas por la prensa convencional sino además de un gran conjunto de opiniones e insumos como enlaces externos que permiten profundizar la información. En el caso de prueba se observa que el 47,57\% corresponden a opiniones generales sobre el tema, el 34,95\% de los tweets se refieren a una noticia, el 11,9\% corresponden a aportes a la discusión y el 5\% corresponden a aportes internacionales.

%Los procesos aplicados a los tweets aportan valor en la medida que permiten extender la comprensión de un hecho noticioso.

Finalmente se concluye que el ejercicio de informarse sobre los hechos noticiosos es fundamental para la generación de opinión ciudadana, una herramienta como la desarrollada en este trabajo, contribuye a esta labor en cuanto facilita el acceso a informaciones difundidas por otras y otros ciudadanos y pone a sus disposición un reporte que no sólo hace referencia a noticias cubiertas por los medios de prensa convencionales sino que también opiniones, puntos de vistas y enlaces de documentación complementarios que permiten profundizar conocimiento sobre el suceso. 

%	\item \textit{Basado en el contenido}: Se refiere a las características relativas al contenido del tweet: novedad del tweet (distancia coseno de términos de los otros tweets que aparecen en el timeline en la última semana) y lo inesperado del tweet (distancia mínima de la distancia coseno de términos con los demás tweets del autor).



\subsection{Consideraciones y discusión sobre las conclusiones}

En la sección \ref{sec:definicion_problema} se profundiza en el interesante y activo debate sobre la influencia real en el usuario y en la construcción de una noticia de los procesos de \emph{gatekeeping}. El grado del impacto en el usuario depende de múltiples variables, entre ellas el ejercicio propio de informarse de cada persona que se dispone a informarse sobre un suceso noticioso. Una visión interesante es la que plantea Ramonet en \cite{fatigaInformarse} referente a que el ejercicio de informarse seriamente requiere esfuerzo y es una ilusión conseguirlo de manera cómoda, como supone la televisión. Es preciso para aprehender toda la complejidad de un suceso recordar los datos fundamentales de un problema, sus antecedentes históricos y su trama social y cultural. Esa misma filosofía sobre informarse, es la que da forma al prototipo al reunir en un mismo espacio comentarios, visiones y opiniones de distintas usuarias y usuarios ordenados y tratados con procesos de \emph{gatekeeping} transparentes además una lista de enlaces externos donde profundizar o complementar puntos de vistas recogidos de múltiples y variadas fuentes. Estos mismos aspectos abren interrogantes sobre la fortaleza de la arquitectura del prototipo desarrollado ¿no es acaso una debilidad importante que solo cuente con una fuente de información como es Twitter? ¿no se expone a caso al \emph{gatekeeping} proporcionado por Twitter?

Aún cuando Twitter, es una de las pocas redes sociales que trabaja constantemente en sus políticas de transparencia ( respecto a las solicitudes de información por parte de los gobiernos)  y plantean abiertamente una postura de transparencia frente a estos asuntos \footnote{``Creemos que el intercambio abierto de información puede tener un impacto global positivo. Para ello, es vital para nosotros (y otros servicios de Internet) para ser transparentes acerca de las solicitudes del gobierno para la información del usuario y de las solicitudes del gobierno para retener contenido de internet, el crecimiento de estas investigaciones pueden tener un efecto negativo grave en la libertad de expresión con implicaciones reales en la privacidad de las personas'' \cite{tweetsStillMustFlow} }, existen precedentes de decisiones comerciales-estratégicas que poseen componentes de censura.

\begin{itemize}
	\item \textbf{Intereses comerciales} como es el caso del bloqueo parcial a 
	Meerkat\footnote{Aplicación que permite realizar streaming de vídeo directamente a los seguidores del usuario en Twitter} competencia directa y de alto grado de utilización, competidora de Periscope, empresa comprada recientemente por Twitter para realizar streaming de vídeo. 
	\item \textbf{Políticas de uso}, como la denegación de acceso a la API para 
	Politwoops\cite{Diplotwoops:Online}, aplicación que hacía visible tweets borrados de políticos en más de 30 países. La cual Twitter justificó de la siguiente manera: "Imagínese: ¿Cómo sería de estresante- o incluso terrorífico-  twittear si fuera irrevocable o inalterable? Ningún usuario es más merecedor de esa capacidad que otro. De hecho, la eliminación de un tweet es una expresión del usuario".
	\item \textbf{Contexto y regulaciones culturales} como es el caso de los sistemas de filtros reactivos (sobre cuentas o tweets) que aplica Twitter para restricciones legales y culturales de los distintos países (escondiendo esos contenidos en sus respectivos países pero dejándolos disponibles en el resto del mundo) \cite{tweetsStillMustFlow}.
\end{itemize}

Considerando lo anterior, con una arquitectura que depende de una sola fuente de información el prototipo efectivamente se expone a filtros de información ejecutados por Twitter (que aún cuando sean transparentados verbalmente, es complejo verificar su real impacto sino no existe posibilidad de acceder al código fuente en cuestión). 

Otro aspecto relevante considerado en este trabajo se refiere al origen del gatekeeping, si corresponde a un motivo operativo o ideológico, durante el desarrollo de este prototipo se verificó operativamente la fluctuación entre estos dos extremos, para conciliar este conflicto se considera que la única solución coherente a esta situación es transparentar los procesos de \emph{gatekeeping} a los usuarios-consumidores de noticias, exponiendo de qué forma actúan y cómo se aplican, de esta manera los usuarios podrán verificar el real efecto que implican en un medio. A modo de metáfora, si tuviéramos la oportunidad de abrir las salas de prensas a miles de auditorías ciudadanas libres, éstas podrían verificar y corroborar las etapas de \emph{gatekeeping} de dicha sala de prensa y validarlas para generar confianza, similar a las prácticas y principios de trasparencia del proyectos \emph{Open source}.
