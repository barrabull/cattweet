Para tener una directriz respecto a qué tópicos de noticias son cubiertos por el medio de prensa objetivo, se realiza un análisis de los tweets de dicho medio.
El proceso se realiza mediante el conteo de la frecuencia de las palabras contenidas en los corpus de los distintos tweets, sin considerar las palabras vacías o stopword\footnote{Stopwords o palabras vacías es el nombre que reciben las palabras sin significado como artículos, pronombres, preposiciones, etc. que son filtradas antes o después del procesamiento de datos en lenguaje natural (texto)} presentes.

\begin{algorithm}
	\caption{Obtención de las palabras más frecuentes del timeline de un conjunto de tweets}\label{ciudadesLeven}
	\begin{algorithmic}[1]
		\Function{getKeywordMedio}{tweets}
		\For{tweet in tweets}
		\For{tweet in tweets}
		\For{palabra in tweet}
		\If{stopwords.not\_in\_array(palabra)}
		\State palabras.push(palabra)
		\State frecPalabras[palabra]++
		\EndIf
		\EndFor
		\EndFor
		\EndFor
		\State return frecPalabras.ordenar()
		\EndFunction
	\end{algorithmic}
\end{algorithm}


%La cuenta de Twitter de AmorTv (\@amor\_tv) al momento de realizar la extracción con más de mil tweets de los cuales se obtuvieron los siguientes resultados:

\begin{table}[H]
	\begin{center}
		\begin{tabular}{| c | c |}
			\hline
			Posición & Palabra \\ \hline
			1 & Estudiantes \\ \hline
			2 & Valparaíso \\ \hline
			3 & Universidad \\ \hline
			4 & Toma \\ \hline
			5 & Sede \\ \hline
			6 & Represión \\ \hline
			7 & Marcha \\ \hline
			8 & Chile \\ \hline
			9 & Concepción \\ \hline
			10 & Casa Central \\ \hline
			11 & Usm  \\ \hline
			12 & Utfsm \\ \hline
			13 & Paro \\ \hline
			14 & Nacional \\ \hline
			15 & Carabineros \\ \hline
			16 & Movimiento \\ \hline
			17 & Pucv \\ \hline
			18 & Asamblea \\ \hline
			19 & Trabajadores \\ \hline
			20 & Secundarios \\ \hline
		\end{tabular}
		\caption{Tabla que muestra en orden descendente las keyword con mayor frecuencia
			en el análisis del timeline de Twitter de Amor TV}
		\label{tab:xyz}
	\end{center}
\end{table}

Se obtiene que las palabras con mayor frecuencia tienen relación con temáticas estudiantiles, manifestaciones, movimientos sociales y los lugares donde estos ocurren, que va acorde a la descripción del medio \cite{amorTV}.

%Tras el posterior análisis con los editores y miembros de amor\_tv sobre las keyword
%que con su conocimiento experto podían considerar, al momento de buscar un evento 
%noticioso agregaron las siguientes:

%	\begin{table}[h]
%	\begin{center}
%	   \begin{tabular}{| c | c |}
%		 \hline
%		   Posición & Palabra \hline
%		   1 & Trabajadores & 
%		   2 & Social & 
%		   3 & Sexismo & 
%		   4 & Santa Maria & 
%		   5 & Protesta & 
%		   6 & Movimiento & 
%		   7 & Movilización & 17 & - \\ \hline
%		   8 & Mapuche & 18 & - \\ \hline
%		   9 & Estudiantil & 19 & - \\ \hline
%		   10 & Ecológico & 20 & - \\ \hline
%		   11 & Confech  \\ \hline
%		   12 & Comunidad \\ \hline
%		   13 & Comunitario \\ \hline
%		   14 & Indígena \\ \hline
%		   15 & Autogestión \\ \hline
%		   16 Medioambiente - \\ \hline
%		\end{tabular}
%		\caption{Tabla que muestra en orden descendente las keyword con mayor frecuencia
%				en el análisis del timeline de Twitter de \@amor\_tv}
%		\label{tab:xyz}
%		 \end{center}
%	 \end{table}

%Las keyword consideradas que representan a mayor cabalidad los eventos noticiosos que
%caracterizan al medio de comunicación de prensa AmorTv, obtenidas de los análisis
%anteriores agregando variaciones de algunas palabras, son representadas gráficamente
%mediante el siguiente wordcloud:

%\begin{figure}[H]
%  \centering
%    \includegraphics[width=0.99\textwidth]{wordcloud.png}
%  \caption{WordCloud de las keyword mas representativas de los eventos noticiosos
%  característicos de AmorTv}
%  \label{fig:storyful}
%\end{figure}
